\documentclass[a4paper, 12pt]{article}
\usepackage{titling}
\usepackage{amsmath}
\usepackage{amssymb}
\usepackage{chngpage}
\usepackage{multirow}
\usepackage{graphicx}
\usepackage{titlesec}
\usepackage{fancyhdr}
\usepackage{chngcntr}
\graphicspath{ {figs/} }
\usepackage{indentfirst}
\usepackage{relsize}
\usepackage[margin=3.7cm]{geometry}
\usepackage{multirow}
\usepackage[table,xcdraw]{xcolor}
\usepackage{hhline}
\usepackage{titletoc}
\usepackage{afterpage}
\usepackage{caption} 
\captionsetup[table]{skip=10pt}

\setlength{\droptitle}{-15em}

\titleformat{\chapter}[display]
{\normalfont\bfseries}{}{0pt}{\LARGE}
\titleformat{\section}[display]
{\normalfont\bfseries}{}{0pt}{\LARGE}
\titleformat{\subsection}[display]
{\normalfont\bfseries}{}{0pt}{\Large}
\titleformat{\subsubsection}[display]
{\normalfont\bfseries}{}{0pt}{\large}
\pagestyle{fancy}
\fancyhf{}
\fancyhead[L]{ELECTRICAL MEASUREMENTS}
\fancyhead[R]{M. de Miguel and Y. Wu}
\fancyfoot[c]{\thepage}

\newcommand\blankpage{%
	\null
	\thispagestyle{empty}%
	\addtocounter{page}{-1}%
	\newpage}


\begin{document}
	\begin{titlepage}
		\centering
		\vfill
		\Large{COMPLUTENSE UNIVERSITY OF MADRID \\ \textbf{FACULTY OF PHYSICAL SCIENCES}}
		\vfill
		\begin{figure}[h!]
			\centering
			\includegraphics[height=7cm]{cumphysics}
		\end{figure}
		\vfill 
		\textbf{\Large{LABORATORY OF ELECTRICITY AND MAGNETISM:}}
		\rule [5pt]{14cm}{2pt}\\
		\Huge{\textbf{ELECTRICAL MEASUREMENTS}} \\
		\rule [8pt]{14cm}{2pt}\\
		\vfill
		\vfill
		\vfill
		\vfill
		
		\large{Mario de Miguel Domínguez and YanRu Wu Jin\\ Bachelor's Degree in Physics, 2\textsuperscript{nd} Year, Group 14\\ Experience date: 9\textsuperscript{th} of February, 2022\\ Delivery date: 16\textsuperscript{th} of February, 2022}
		\vfill
		\vfill
		\vfill
		\vfill
		
		\afterpage{\blankpage}
	\end{titlepage}
	
	\makeatletter
	\thispagestyle{empty}
	\addtocounter{page}{-1}
	\let\latexl@section\l@section
	\def\l@section#1#2{\begingroup\let\numberline\@gobble\latexl@section{#1}{#2}\endgroup}
	\let\latexl@subsection\l@subsection
	\def\l@subsection#1#2{\begingroup\let\numberline\@gobble\latexl@subsection{#1}{#2}\endgroup}
	\let\latexl@subsubsection\l@subsubsection
	\def\l@subsubsection#1#2{\begingroup\let\numberline\@gobble\latexl@subsubsection{#1}{#2}\endgroup}
	\makeatother
	\tableofcontents	
	\thispagestyle{empty}
	
	\afterpage{\blankpage}
	\newpage
\section{1 Introduction}
This assignment is aimed at gaining insight into both the usage of multimeters and the analysis of simple direct-current (DC) and alternating-current (AC) circuits with capacitors, coils and resistors. \\

To achieve this, several experiences will be carried out involving two different voltage sources, one being a DC source and the other an AC supply of variable amplitude and a frequency $\nu = 50$ s\textsuperscript{-1}. 
\subsection{1.1 Ohm's law for direct-current circuitry}
Ohm's law establishes a linear relationship between the current $I$ flowing through a circuit and the potential drop $V$ among its terminals. The ratio of these two quantities is known as the resistance of the circuit $R$. Ohm's law then reads
\begin{equation}\label{ohmslaw}
	R = \frac{V}{I}
\end{equation}
\subsection{1.2 Ohm's law for alternating-current circuitry}
Alternating current sources offer signals of the form $V(t) = V_0 \cos (\omega t)$, with $V_0$ as the maximum amplitude and $\omega$ as the angular frequency ($\omega = 2\pi\nu$) of the signal. Hence, when dealing with alternating current calculations are normally evaluated through the effective voltage, which may be defined as 
\begin{equation}\label{vef}
	V_{ef} = \sqrt{\nu \int_0^T [f(t)]^2 dt},
\end{equation}
where $\nu$ is the frequency of the signal. Particularising for a sinusoidal signal $V_{ef} = \frac{V_0}{\sqrt 2}$.\textsuperscript{[1]}\\

The current $I$ will have the same angular frequency $\omega$ but a phase difference of $\delta$ with respect to the voltage signal:
\begin{equation}\label{vi}
	V = V_0e^{i\omega t},
\end{equation}
\begin{equation}\label{ii}
	I = I_0e^{i(\omega t - \delta)}.
\end{equation}
Ohm's law shall now be recast as
\begin{equation}\label{ohm2}
	Z = \frac{V}{I},
\end{equation}
where $ Z \in \mathbb{C} $ is the impedance of the circuit. Note that
\begin{equation}\label{z}
	Z = |Z|e^{i\delta},
\end{equation}
with $|Z| = \frac{|V|}{|I|}$ and $\delta =  tan^{-1} \left(\frac{Im(Z)}{Re(Z)}\right)$.
\section{2 Materials and Experimental Method}
\subsection{2.1 Materials}
\begin{itemize}
	\item Multimeters (voltmeter and ammeter)
	\item AC source
	\item DC source
	\item Circuit board
	\item Connecting wires
	\item Resistors (100 $\Omega$ and 470 $\Omega$)
	\item Capacitors (1.0 $\mu F$ and  4.7 $\mu F$)
	\item Connecting bridges
	\item Inductor coil
\end{itemize}
\subsection{2.2 Experimental Method}
In this experiment three different setups with alternating and direct current are to be studied.\\

First, the multimeters must be correctly set, one to be able to measure the voltage (voltmeter) and other to be able to measure the electric current (ampere meter). The voltmeter must be attached to the circuit board through the voltmeter (V$/\Omega$) and common (COM) connections, while the ammeter must present the low current (mA) and common (COM) connections attached to the circuit.\\

It is important to connect the multimeters appropriately.The voltmeter shall be connected in parallel with the component whose voltage drop is to be measured, for according to Kirchhoff's laws, the voltage drop through that component must be equal to that in the voltmeter. Furthermore, its internal resistance is high enough ($\longrightarrow \infty$) not to alter the current passing through the component. \\

The ammeter must be attached to the board in series with the circuit component of which the current is to be measured. It is important to never connect the ammeter in parallel: its resistance is high enough not to alter the current flow through the component ($\longrightarrow 0$), and Kirchhoff's laws state that the voltage drop through the component and the ammeter must be the same if connected in parallel, meaning this the current in the ammeter would go to infinity. This is highly likely to blow the fuse of the ammeter.

\section{3 Experimental Results}
The measurements taken at the lab shall be hereby displayed. For diagrams of the circuit setups or further information about the used expressions for the uncertainty estimations please refer to sections 1 and 3 in the appendix, respectively. For all AC circuits the $V_{total} \equiv V_{efficient} \approx 10$V. This means $V_{peak} = \sqrt 2 V_{total} \approx 14.14$V.
\subsection{3.1.1 AC circuit with two resistors in series}
The first built circuit consists of two resistors in series, the ammeter (connected in series) and the voltmeter (in parallel with respect to one of the resistors at a time). Therefore the relationships
\begin{align*}\label{resser}
	I_{total} &= I_1 = I_2 \mbox{ and} \\
	V_{total} &= V_1 + V_2,
\end{align*}
which are deduced from Kirchhoff's laws, must be fulfilled. Note that $I_i$ corresponds to the current passing through the $i$-th resistor, $V_i$ each voltage drop and $R_i$ its resistance value. Since only resistors conform the circuit,
\begin{equation}\label{resserz}
	R_{eq} = Z_{total}.
\end{equation}\\

If the previous relationships are combined with Ohm's law, it can be concluded that
\begin{equation}\label{req}
	Z_{total} = \sum_{i} Z_i \implies R_{eq} = \frac{V_{total}}{I_{total}}= R_1 + R_2
\end{equation}\\

Table 1 displays the measurements for circuit 1.1.
\begin{table}[hbt!]
	\centering
	\begin{tabular}{|c|c|c|}
		\hline
		& \textbf{Theoretical values} & \textbf{Experimental values} \\
		\hline
		\textbf{$V_{1}$ (V)} & 1.75 & 1.73 $\pm$ 0.01 \\
		\textbf{$V_{2}$ (V)} & 8.25 & 8.18 $\pm$ 0.01 \\
		\textbf{$V_{total}$ (V)} & 10 & 9.91 $\pm$ 0.01 \\
		\hline
		\textbf{$I_{total}$ (mA)} & 17.5 & 17.4 $\pm$ 0.1 \\
		\hline
		\textbf{$R_{1}$ ($\Omega$)} & 100 & 99.43 $\pm$ 0.81 \\
		\textbf{$R_{2}$ ($\Omega$)} & 470 & 470.1 $\pm$ 2.8 \\
		\textbf{$R_{eq}$ ($\Omega$)} & 570 & 569.5 $\pm$ 3.3 \\
	
		\hline
	\end{tabular}
	\caption{Measurements for circuit 1.1: AC, two resistors in series}
\end{table}\\

It can be checked that indeed $V_1 + V_2 = 9.91 $ V and $R_1 + R_2 = 569.53 $ $\Omega$.
\subsection{3.1.2 AC circuit with two resistors in parallel}
This setup remains similar to the previous one, with the exception that the resistors are now disposed in parallel, rather than in series. Now, according to Kirchhoff's laws,
\begin{align*}\label{respar}
	I_{total} &= I_1 + I_2, \\
	V_{total} &= V_1 = V_2. \\
\end{align*}\\

Considering again equation \ref{resserz} and combining the previous Kirchhoff's relations with Ohm's law, it can be reached the conclusion that
\begin{equation}\label{ztotrespar}
	Z_{total}^{-1} = \sum_{i} Z_i^{-1} \implies R_{eq}^{-1} = \frac{I_{total}}{V_{total}} = R_1^{-1} + R_2^{-1}.
\end{equation}\\

The measurements, gathered in Table 2 have been taken with the two same resistors as before.
\begin{table}[hbt!]
	\centering
	\begin{tabular}{|c|c|c|}
		\hline
		& \textbf{Theoretical values} & \textbf{Experimental values} \\
		\hline
		\textbf{$V_{1}$ (V)} & 10 & 9.98 $\pm$ 0.01 \\
		\textbf{$V_{2}$ (V)} & 10 & 9.90 $\pm$ 0.01 \\
		\textbf{$V_{total}$ (V)} & 10 & 10.00 $\pm$ 0.01\\
		\hline
		\textbf{$I_{1}$ (mA)} & 100  & 100.5 $\pm$ 0.1 \\
		\textbf{$I_{2}$ (mA)} & 21.3 & 21.0 $\pm$ 0.1 \\
		\textbf{$I_{total}$ (mA)} & 121.3 & 122.9 $\pm$ 0.1 \\
		\hline
		\textbf{$R_{1}$ ($\Omega$)} & 100 & 99.303 $\pm$ 0.081 \\
		\textbf{$R_{2}$ ($\Omega$)} & 470 & 471.429 $\pm$ 0.017 \\
		\textbf{$R_{eq}$ ($\Omega$)} & 82.46 & 81.37 $\pm$ 0.10 \\
		
		\hline
	\end{tabular}
	\caption{Measurements for circuit 1.2: AC, two resistors in parallel}
\end{table}\\

It can be checked that $I_1 + I_2 = 121.5 $ mA, which lies closely to both the measured and the theoretical values for $I_{total}$. Furthermore, $(1/R_{1} + 1/R_2)^{-1} = 82.02$ $\Omega$, which is also close to the $R_{eq}$ results.

\newpage
\subsection{3.2.1 AC circuit with two capacitors in series}
Capacitors undergo cycles of charging and discharging when provided with AC supplies. Consequently, it is not very strange thus that their impedance does not depend only on their capacitance, but also on the current oscillation frequency (50 s\textsuperscript{-1}). \\

Since the circuit is built in series, it must obey the following relations:
\begin{align*}
	I_{total} &= I_1 = I_2, \\
	V_{total} &= V_1 + V_2.\
\end{align*}\\

The impedance of a capacitor haves only an imaginary part and can be expressed as
\begin{equation}\label{capz}
	Z = -\frac{i}{\omega C},
\end{equation}
with $C$ as the capacitance of the condensator and $\omega$ as the angular frequency of the current. In other words,
\begin{equation}\label{capzabs}
	|Z| = \frac{1}{\omega C}.
\end{equation}\\

It can now be stated that
\begin{equation}\label{key}
	|Z_{total}| = \frac{|V_{total}|}{|I_{total}|}= \frac{1}{\omega C_{eq}}
\end{equation}
where, since for series circuitry $Z_{total} = \sum_i Z_i$,
\begin{equation}\label{ceq1}
	C_{eq}^{-1} = C_1^{-1} + C_2^{-1}.
\end{equation}
The measurements for circuit 2.1 are displayed in Table 3. 
\begin{table}[hbt!]
	\centering
	\begin{tabular}{|c|c|c|}
		\hline
		& \textbf{Theoretical values} & \textbf{Experimental values} \\
		\hline
		\textbf{$V_{1}$ (V)} & 8.25 & 8.26 $\pm$ 0.01 \\
		\textbf{$V_{2}$ (V)} & 1.75 & 1.77 $\pm$ 0.01 \\
		\textbf{$V_{total}$ (V)} & 10 & 10.02 $\pm$ 0.01\\
		\hline
		\textbf{$I_{1}$ (mA)} & 2.59  & 2.54 $\pm$ 0.01 \\
		\textbf{$I_{2}$ (mA)} & 2.59 & 2.54 $\pm$ 0.01 \\
		\textbf{$I_{total}$ (mA)} & 2.59 & 2.54 $\pm$ 0.01 \\
		\hline
		\textbf{$C_{1}$ ($\mu$F)} & 1 & 0.9788 $\pm$ 0.0031 \\
		\textbf{$C_{2}$ ($\mu$F)} & 4.7 & 4.576 $\pm$ 0.032 \\
		\textbf{$C_{eq}$ ($\mu$F)} & 0.8246 & 0.8063 $\pm$ 0.0033 \\
		\hline
		\textbf{$|Z_{1}|$ (k$\Omega$)} & 3.183 & 3.252 $\pm$ 0.013 \\
		\textbf{$|Z_{2}|$ (k$\Omega$)} & 0.6773 & 0.6967 $\pm$ 0.0048 \\
		\textbf{$|Z_{total}|$ (k$\Omega$)} & 3.860 & 	3.947 $\pm$ 0.016 \\
		\hline
	\end{tabular}
	\caption{Measurements for circuit 2.1: AC, two capacitors in series}
\end{table}\\

The impedance values have been expressed in terms of the impedance moduli to easen their reading. Note that for any capacitor or combination of capacitors $Z = -i|Z|$.\\

The usual checks can be performed: 
$|Z_1| + |Z_2| = 3.945 $ k$\Omega$ and $(1/C_1 + 1/C_2)^{-1} = 8.063 $ $\mu$F

\subsection{3.2.2 AC circuit with two capacitors in parallel}
Capacitors in parallel must also satisfy Kirchhoff's laws, what leads to the same relationships of current and voltage as resistors in parallel:
\begin{align*}
	I_{total} &= I_1 + I_2, \\
	V_{total} &= V_1 = V_2.
\end{align*}
And now the total impedance 
\begin{equation}\label{ztotcappar}
	|Z_{tot}| = \frac{|V_{total}|}{|I_{total}|} = \frac{1}{\omega C_{eq}}.
\end{equation}
Since for parallel circuits $Z_{total}^{-1} = \sum_i Z_i^{-1}$,
\begin{equation}\label{ceq2}
	C_{eq} = C_1 + C_2.
\end{equation}
\begin{table}[hbt!]
	\centering
	\begin{tabular}{|c|c|c|}
		\hline
		& \textbf{Theoretical values} & \textbf{Experimental values} \\
		\hline
		\textbf{$V_{1}$ (V)} & 10 & 10.00 $\pm$ 0.01 \\
		\textbf{$V_{2}$ (V)} & 10 & 10.00 $\pm$ 0.01 \\
		\textbf{$V_{total}$ (V)} & 10 & 10.00 $\pm$ 0.01\\
		\hline
		\textbf{$I_{1}$ (mA)} & 3.14 & 3.08 $\pm$ 0.01 \\
		\textbf{$I_{2}$ (mA)} & 14.77 & 14.61 $\pm$ 0.01 \\
		\textbf{$I_{total}$ (mA)} & 17.91 & 17.77 $\pm$ 0.01 \\
		\hline
		\textbf{$C_{1}$ ($\mu$F)} & 1 & 0.9804 $\pm$ 0.0056 \\
		\textbf{$C_{2}$ ($\mu$F)} & 4.7 & 4.6505 $\pm$ 0.0033 \\
		\textbf{$C_{eq}$ ($\mu$F)} & 5.7 & 5.6309 $\pm$ 0.0064 \\
		\hline
		\textbf{$|Z_{1}|$ (k$\Omega$)} & 3.183 & 3.247 $\pm$ 0.011 \\
		\textbf{$|Z_{2}|$ (k$\Omega$)} & 0.67726 & 0.6845 $\pm$ 0.00065* \\
		\textbf{$|Z_{total}|$ (k$\Omega$)} & 0.55843 & 0.56529 $\pm$ 0.00083* \\
		\hline
	\end{tabular}
	\caption{Measurements for circuit 2.2: AC, two capacitors in parallel}
\end{table}\\

It can be seen that $(1/|Z_1| + 1/|Z_2|)^{-1} =  0.565 $ k$\Omega$ and $C_1 + C_2 = 5.631 $ $\mu$F, so it is seen that the pertinent relations also hold for this circuit.
\subsection{3.3.1 DC circuit with an inducting coil}
Inductors have a complex impedance with both a real and an imaginary part. The real part consists of their resistance $R$ while the imaginary part is related to their self-induction coefficient $L$. The expression for the impedance of an inductor reads
\begin{equation}\label{zldc}
	Z_L = Re(Z_L) + iIm(Z_L) = R_L + i\omega L.
\end{equation}\\

 Since for direct-current circuitry $\omega = 0$ rad/s, 
 \begin{equation}\label{rl}
 	Z_L = |Z_L| = R_L.
 \end{equation}\\

The built circuit includes a 100 $\Omega$ resistor and a 1000-turn coil connected in series, it allowing to calculate the resistance of the coil. Using Ohm's law:
\begin{equation}\label{zlohm}
	Z_L = R_L = \frac{V_L}{I_{total}},
\end{equation} 
where $V_L$ is the voltage drop through the coil and $I_{tot}$ the total current in the circuit.
Table 5 displays the measurements of the current and voltage drop in the coil so the real part of its impedance $R_L$ can be estimated. 
\begin{table}[hbt!]
	\centering
	\begin{tabular}{|c|c|c|}
		\hline
		& \textbf{Theoretical values} & \textbf{Experimental values} \\
		\hline
		\textbf{$V_{L}$ (V)} & 10 & 1.56 $\pm$ 0.01 \\
		\textbf{$V_{total}$ (V)} & 10 & 10.01 $\pm$ 0.01\\
		\hline
		\textbf{$I_{total}$ (mA)} & & 84.4 $\pm$ 0.1 \\
		\hline
		\textbf{$R_{L}$ ($\Omega$)} &  & 18.48 $\pm$ 0.12 \\
		\hline
	\end{tabular}
	\caption{Measurements for circuit 3.1: DC, coil and resistor in series}
\end{table}

\subsection{3.3.2 AC circuit with an inducting coil}
If the same circuit of section 3.3.1 is provided with an alternating signal, the inductor undergoes self induction phenomena owing to the changing current intensity through the coil. The impedance of the inductor is now given by expression \ref{zldc}, and its absolute value reads:
\begin{equation}\label{abszl}
	|Z_L| = \frac{|V_L|}{|I_{total}|} = \sqrt{Z_L \cdot Z_L^\dagger} = \sqrt{R_L^2 + \omega^2L^2}
\end{equation}\\

This can be used to compute the imaginary part of the impedance $Z_L$ as
\begin{equation}\label{imzl}
	Im(Z_L) = \sqrt{|Z_L| - R_L^2}.
\end{equation}
taking the $R_L$ value as that estimated in the previous section.
\begin{table}[hbt!]
	\centering
	\begin{tabular}{|c|c|c|}
		\hline
		& \textbf{Theoretical values} & \textbf{Experimental values} \\
		\hline
		\textbf{$V_{L}$ (V)} & 10 & 3.00 $\pm$ 0.01 \\
		\textbf{$V_{total}$ (V)} & 10 & 10.02 $\pm$ 0.01\\
		\hline
		\textbf{$I_{total}$ (mA)} & & 80.8 $\pm$ 0.1 \\
		\hline
		\textbf{$|Z_{L}|$ ($\Omega$)} &  & 37.13 $\pm$ 0.13 \\
		\hline
	\end{tabular}
	\caption{Measurements for circuit 3.2: AC, coil and resistor in series}
\end{table}\\

Known both the values of $|Z_L|$ and $R_L$, the imaginary part $\omega L$ can be computed as 
\begin{equation}\label{imagniaryz}
	Im(Z_L) = \sqrt{|Z_L|^2 - R_L^2}.
\end{equation}\\

Using equation \ref{imagniaryz} and the data from tables 5 and 6, it is found out that
\begin{equation*}
	Im(Z_L) = 32.20 \pm 0.17 \mbox{ } \Omega. 
\end{equation*}
So
\begin{equation*}
		Z_L = (18.48 \pm 0.12) + i(32.20 \pm 0.17) \mbox{ } \Omega
\end{equation*}
and through expression \ref{z}:
\begin{equation*}
	Z_L = (37.13 \pm 0.13)e^{i(1.050 \pm 0.042)} \mbox{ } \Omega. 
\end{equation*}
\section{4 Conclusion}
It can be seen that few experimental results were compatible with the theoretical ones. Nonetheless, they all remain fairly close to them, being the majority of deviations of around 1.2\% and the highest of about 2.8\%. The usual checks for the Kirchhoff's relations of every circuit are satisfied by the experimental values, again with some unobtrusive deviations.\\

It must be remarked there have been found some conspicuous anomalies when computing the uncertainties (marked with an asterisk in the tables), as the $Z_2$ and $Z_{total}$ in theparallel setup were significantly more precise as those for the 1 $\mu$F capacitor. This can be explained with the high dependency on the intensity going through each presented by the equations of the capacitor impedance uncertainties (in this case, 39 and 40). \\

It can also be seen that the voltage measurements are not exactly equal for each branch of the parallel-resistor circuit. This could be due to the state of one end of the wires used to join the voltmetre and the circuit, which was very sensitive to the way it was bent when plugged to the circuit. This cable was replaced for the rest of the measurements.\\

In the case one wants to calculate the value of the coil inductance, this could be easily done known the imaginary part of $Z_L$. Using the obtained value for $Im(Z_L)$:
\begin{equation*}
	L = \frac{Im(Z_L)}{\omega} = 0.1 \mbox{ H.}
\end{equation*}\\

Furthermore, if the dimensions of the inductor were known, the magnetic permeability $\mu_0$ could be calculated as
\begin{equation}\label{mu0}
	\mu_0 = \frac{Ll}{N^2 A},
\end{equation}
where $l$ is the length of the solenoid, $N$ the number of turns and A the area within the turns; and checked to be equal to the theoretical value $\mu_0 = 4\pi \cdot 10^{-7}$ H/m.\textsuperscript{[2]}\\

To conclude with, the experiment results may be said to be satisfactory, for they are sensible and coherent with the expected values.
\section{References}
[1] “Experimental Guide: Electrical Measurements.” Complutense University of Madrid, 2021. \\

[2] Young, H. D., Freedman, R. A., \& Ford, A. L. (2020). \textit{University physics with modern physics}. Pearson Education Limited. 
\newpage


\section{Appendix}
\subsection{A.1 Circuitry}
This section displays diagrams of the used circuits. Note that the ammeter and the voltmeter may have been moved in order to perform all the measurements. 
\begin{figure}[hbt!]
	\centering
	\includegraphics{circuit11}
	\caption{Circuit 1.1: AC, resistors in series}
\end{figure}
\begin{figure}[hbt!]
	\centering
	\includegraphics{circuit12}
	\caption{Circuit 1.2: AC, resistors in parallel}
\end{figure}
\begin{figure}[hbt!]
	\centering
	\includegraphics[height=5.5cm]{circuit21}
	\caption{Circuit 2.1: AC, capacitors in series}
\end{figure}
\begin{figure}[hbt!]
	\centering
	\includegraphics[height=6.5cm]{circuit22}
	\caption{Circuit 2.2: AC, capacitors in parallel}
\end{figure}
\begin{figure}[hbt!]
	\centering
	\includegraphics[height=5.5cm]{circuit31}
	\caption{Circuit 3.1: DC, coil and resistor in series}
\end{figure}
\begin{figure}[hbt!]
	\centering
	\includegraphics{circuit32}
	\caption{Circuit 3.2: AC, coil and resistor in series}
\end{figure}
\subsection{A.2 Presentation of results}
Experimental results have been rounded considering up to two significant figures in their uncertainties. Tabular of theoretical values have been presented without uncertainties, rounded to the same decimal figures as their corresponding experimental results unless exact.\\

Measurements have been displayed in easy-to-read units, these not necessarily being part of the International System of Units.
\subsection{A.3 Error estimation}
\subsubsection{A.3.1 Direct measurements}
Direct measurement uncertainties have been set to be the precision of the used devices i.e. the ammeter and the voltmeter. Current intensity uncertainties vary in terms of the effective current order of magnitude, for this would determine in what scale would they be measured in the ammeter.
\subsubsection{A.3.2 Indirect measurements}
The uncertainty formulas for indirect uncertainties have been found through the partial derivative method.\\

\textbf{Circuit 1.1}
\begin{align}
	\Delta R_1 &= \sqrt{\left(\frac{\Delta V_1}{I_{tot}}\right)^2 + \left(\frac{V_1\Delta I_{tot}}{I_{tot}^2}\right)^2} \\
	\Delta R_2 &= \sqrt{\left(\frac{\Delta V_2}{I_{tot}}\right)^2 + \left(\frac{V_2\Delta I_{tot}}{I_{tot}^2}\right)^2} \\
	\Delta R_{tot} &= \sqrt{\left(\frac{\Delta V_{tot}}{I_{tot}}\right)^2 + \left(\frac{V_{tot}\Delta I_{tot}}{I_{tot}^2}\right)^2} 
\end{align}

\textbf{Circuit 1.2}
\begin{align}
	\Delta R_1 &= \sqrt{\left(\frac{\Delta V_{tot}}{I_{1}}\right)^2 + \left(\frac{V_{tot}\Delta I_{1}}{I_{1}^2}\right)^2} \\
	\Delta R_2 &= \sqrt{\left(\frac{\Delta V_{tot}}{I_{2}}\right)^2 + \left(\frac{V_{tot}\Delta I_{2}}{I_{2}^2}\right)^2} \\
	\Delta R_{tot} &= \sqrt{\left(\frac{\Delta V_{tot}}{I_{tot}}\right)^2 + \left(\frac{V_{tot}\Delta I_{tot}}{I_{tot}^2}\right)^2} 
\end{align}

\textbf{Circuit 2.1}
\begin{align}
	\Delta C_1 = \frac{\Delta |Z_1|}{\omega |Z_1|^2}\\
	\Delta C_2 = \frac{\Delta |Z_1|}{\omega |Z_2|^2}\\
	\Delta C_3 = \frac{\Delta |Z_1|}{\omega |Z_3|^2}
\end{align}\\

\begin{align}
	\Delta |Z_1| &= \sqrt{\left(\frac{\Delta V_1}{I_{tot}}\right)^2 + \left(\frac{V_1\Delta I_{tot}}{I_{tot}^2}\right)^2} \\
	\Delta |Z_2| &= \sqrt{\left(\frac{\Delta V_2}{I_{tot}}\right)^2 + \left(\frac{V_2\Delta I_{tot}}{I_{tot}^2}\right)^2} \\
	\Delta |Z_{tot}| &= \sqrt{\left(\frac{\Delta V_{tot}}{I_{tot}}\right)^2 + \left(\frac{V_{tot}\Delta I_{tot}}{I_{tot}^2}\right)^2} 
\end{align}

\textbf{Circuit 2.2}
\begin{align}
	\Delta C_1 = \frac{\Delta |Z_1|}{\omega |Z_1|^2}\\
	\Delta C_2 = \frac{\Delta |Z_1|}{\omega |Z_2|^2}\\
	\Delta C_3 = \frac{\Delta |Z_1|}{\omega |Z_3|^2}
\end{align}\\

\begin{align}
	\Delta |Z_1| &= \sqrt{\left(\frac{\Delta V_{tot}}{I_{1}}\right)^2 + \left(\frac{V_{tot}\Delta I_{1}}{I_{1}^2}\right)^2} \\
	\Delta |Z_2| &= \sqrt{\left(\frac{\Delta V_{tot}}{I_{2}}\right)^2 + \left(\frac{V_{tot}\Delta I_{2}}{I_{2}^2}\right)^2} \\
	\Delta |Z_{tot}| &= \sqrt{\left(\frac{\Delta V_{tot}}{I_{tot}}\right)^2 + \left(\frac{V_{tot}\Delta I_{tot}}{I_{tot}^2}\right)^2} 
\end{align}

\textbf{Circuit 3.1}
	\begin{equation}
		\Delta R_{L} = \sqrt{\left(\frac{\Delta V_{tot}}{I_{tot}}\right)^2 + \left(\frac{V_{tot}\Delta I_{tot}}{I_{tot}^2}\right)^2} 
	\end{equation}
	
\textbf{Circuit 3.2}
	\begin{equation}
		\Delta |Z_{L}| = \sqrt{\left(\frac{\Delta V_{tot}}{I_{tot}}\right)^2 + \left(\frac{V_{tot}\Delta I_{tot}}{I_{tot}^2}\right)^2} 
	\end{equation}
	\begin{equation}
		\Delta Im (Z_L) = \sqrt{\left(\frac{|Z_L|\Delta |Z_L|}{\sqrt{|Z_L|^2 - R_L^2}}\right)^2 + \left(\frac{R_L\Delta R_L}{\sqrt{|Z_L^2 - R_L^2|}}\right)^2}
	\end{equation}
	\begin{equation}\label{deltadelta}
	\Delta \delta = \sqrt{\left(\frac{\Delta Im(Z_L)}{R_L + \frac{Im(Z_L)^2}{R_L}}\right)^2 + \left(\frac{Im (Z_L)\Delta R_L}{R_L^2 + Im(Z_L)^2}\right)^2}	
	\end{equation}
\end{document}